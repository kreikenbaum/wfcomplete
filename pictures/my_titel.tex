\documentclass[a4paper,10pt]{article}
% Wenn Sie eine andere Dokumentenklasse benutzen, kann sich das Layout verschieben 
%({scrreprt} verursacht beispielsweise einen ungewollten Seitenumbruch).
% Sie können in diesem Fall versuchen, die Abstände (\vspace, siehe unten) anzupassen 
% oder Sie compilieren das Titelblatt einzeln mit der Dokumentenklasse ``article''.
\usepackage{graphicx}
\usepackage{float}
\usepackage[T1]{fontenc}
\usepackage[utf8]{inputenc}

\begin{document}
\thispagestyle{empty}

\hspace{20cm}
\vspace{-2cm}

\begin{figure}[H] \hspace{11cm}
\includegraphics[width=3.2 cm]{HU_Logo}
\end{figure}

\begin{center}
  \vspace{0.5 cm}
  \huge{\bf Selective Cover Traffic} \\ % Hier fuegen Sie den Titel Ihrer Arbeit ein.
  \vspace{1.5cm}
  \LARGE  Diplomarbeit \\ % Geben Sie anstelle der Punkte an, ob es sich um eine
                % Diplomarbeit, eine Masterarbeit oder eine Bachelorarbeit handelt.
  \vspace{1cm}
  \Large zur Erlangung des akademischen Grades \\
  Diplominformatiker \\ % Bitte tragen Sie hier anstelle der Punkte ein:
         % Diplominformatiker(in),
         % Bachelor of Arts (B. A.),
         % Bachelor of Science (B. Sc.),
         % Master of Education (M. Ed.) oder
         % Master of Science (M. Sc.).
  \vspace{2cm}
  {\large
    \bf{
      \scshape
      Humboldt-Universit\"at zu Berlin \\
      Mathematisch-Naturwissenschaftliche Fakult\"at II \\
      Institut f\"ur Informatik\\
    }
  } 
  % \normalfont
\end{center}
\vspace {5 cm}% gegebenenfalls kleiner, falls der Titel der Arbeit sehr lang sein sollte
%{3.2 cm} bei Verwendung von scrreprt, gegebenenfalls kleiner, falls der Titel der Arbeit sehr lang sein sollte
{\large
  \begin{tabular}{llll}
    eingereicht von:    & Michael Kreikenbaum && \\ % Bitte Vor- und Nachnamen anstelle der Punkte eintragen.
    geboren am:         & 13.09.1981 && \\
    in:                 & Northeim && \\
    &&&\\
    Gutachter:          & Prof. Dr. Konrad Rieck (Universität Braunschweig) && \\
		        & Prof. Dr. Marius Kloft && \\% Bitte Namen der Gutachter(innen) anstelle der Punkte eintragen
				 % bei zwei männlichen Gutachtern kann das (innen) weggestrichen werden
    &&&\\
    eingereicht am:     & \dots\dots \\ % Bitte lassen Sie
                                    % diese beiden Felder leer.
                                    % Loeschen Sie ggf. das letzte Feld, wenn
                                    % Sie Ihre Arbeit laut Pruefungsordnung nicht
                                    % verteidigen muessen.
  \end{tabular}
}
\end{document}
